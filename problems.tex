\section{Задачи}
\subsection*{Задача 1}
На территории района имеется 6 магазинов.
Необходимо определить место расположения распределительного склада и нанести его на чертеж.
Исходные данные: координаты магазинов, грузооборот (таблица \ref{coord}).

\begin{table}[h!]
	\small
	\centering
	\caption{Координаты, грузооборот магазинов}
	\label{coord}
	\setlength{\extrarowheight}{1mm}
	\begin{tabularx}{\textwidth}{|p{3.73cm}|K{1.7cm}|K{1.7cm}|K{1.7cm}|K{1.7cm}|K{1.7cm}|K{1.7cm}|}
		\hline
		№ магазина      & 1  & 2  & 3  & 4  & 5  & 6  \\ \hline
		Координаты по Х & 15 & 25 & 40 & 50 & 80 & 90 \\ \hline
		Координаты по У & 35 & 45 & 70 & 80 & 40 & 60 \\ \hline
		Грузооборот, тонн/мес &10&35&25&40&45&60 \\ \hline
	\end{tabularx}
\end{table}

Координаты центра тяжести грузовых потоков ($X_{\text{склад}}, Y_{\text{склад}}$), то есть точки, в которых может быть размещен распределительный склад, определяются по формулам:
\[X_{\text{склад}}= \dfrac{\sum\limits_{i=1}^{n} \text{Г}_{i} \cdot X_{i}}{\sum\limits_{i=1}^{n} \text{Г}_{i}}, \]
\[Y_{\text{склад}}= \dfrac{\sum\limits_{i=1}^{n} \text{Г}_{i} \cdot Y_{i}}{\sum\limits_{i=1}^{n} \text{Г}_{i}}, \]
где:\\ $\text{Г}_{i}$ --- грузооборот i-го потребителя,
\\
$X_{i}, Y_{i}$ --- координаты i-го потребителя,
\\
$n $-- число потребителей.

Изобразим данные графически и рассчитаем координаты центра тяжести:
\begin{figure}[h]
	\centering
	\includegraphics[width=1\linewidth]{problem1}
	\caption{Координаты магазинов с указанным грузооборотом}
	\label{fig:problem1}
\end{figure}

\[X_{\text{склад}}= \dfrac{10 \cdot 15 + 35\cdot25+25\cdot40+40\cdot50+45\cdot80+60\cdot90}{10+35+25+40+45+60} = 60,58, \]
\[Y_{\text{склад}}= \dfrac{10\cdot35+35\cdot45+25\cdot70+40\cdot80+45\cdot40+60\cdot60}{10+35+25+40+45+60} = 57,09. \]

Выполненный расчет показывает, что склад нужно разместить в точке с координатами Х = 60,58; Y = 57,09.

На реальной местности точка оптимального размещения склада обычно не совпадает с найденным центром тяжести грузопотоков, но, как правило, находится где-то недалеко.
Подобрать приемлемое место для склада позволит последующий анализ возможных мест размещения в окрестностях найденного центра тяжести.

\subsection*{Задача 2}
Определить уровень логистического сервиса на основе следующих данных.
Общий список услуг, которые могут быть оказаны фирмой в процессе поставки товаров, а также время, необходимое для оказания каждой отдельной услуги (чел./час.).

\begin{table}[]
	\small
	\centering
	\caption{Исходные данные}
	\label{my-label}
	\setlength{\extrarowheight}{1mm}
	\begin{tabularx}{\textwidth}{|p{4.2cm}|K{0.8cm}|K{0.8cm}|K{0.8cm}|K{0.8cm}|K{0.8cm}|K{0.8cm}|K{0.8cm}|K{0.8cm}|K{0.8cm}|K{0.8cm}|}
		\hline
		№ услуги                                           & 1   & 2   & 3   & 4   & 5   & 6   & 7   & 8   & 9   & 10  \\ \hline
		Время, необходимое для оказания услуги (чел./час.) & 1.5 & 2.3 & 1.4 & 0.8 & 4.0 & 3.8 & 0.9 & 1.2 & 1.6 & 0.5 \\ \hline
		Фактически фирма оказывает услуги&&&X&&X&X&&X&X&X \\ \hline
	\end{tabularx}
\end{table}

Уровень логистического сервиса --- величина, характеризующая логистический сервис и равная отношению практических и теоретических (оптимальных) значений показателей количества и качества логистических услуг.
Эта величина позволяет оценить систему сервиса предприятия с позиции поставщика и потребителя услуг.

Расчет уровня логистического сервиса выполняется следующим образом:
\[ \eta = \dfrac{m}{M} \cdot 100\%, \]
где:\\
$\eta$ --- уровень логистического сервиса;\\
$m$ --- количественная оценка практического объема оказываемых услуг;\\
$М$ --- количественная оценка теоретически возможного объема оказываемых услуг.

Уровень логистического сервиса можно также оценить отношением фактического времени оказания услуг и теоретического времени, необходимого на оказание всего комплекса существующих на данном предприятии услуг:
\[ \eta = \dfrac{\sum\limits_{i=1}^{n} t_i}{\sum\limits_{i=1}^{N} t_i} \cdot 100\%,\]
где: \\
$\eta$ --- уровень логистического сервиса;
\\
$t_i$ --- время, затрачиваемое на выполнение i-й услуги;
\\
$n$ --- фактическое количество оказанных услуг;
\\
$N$ --- количество существующих услуг.

Таким образом, 
\\
$\sum\limits_{i=1}^{n} t_i$ --- суммарное время, фактически затрачиваемое на оказание услуг, а 
\\
$\sum\limits_{i=1}^{N} t_i$ --- время, которое теоретически может быть затрачено на выполнение всего комплекса возможных услуг.

Так как фирма оказывает услуги №3, №5, №6, №8, №9, №10, то
\[ \eta = \dfrac{1,4+4,0+3,8+1,2+1,6+0,5}{1,5+2,3+1,4+0,8+4,0+3,8+0,9+1,2+1,6+0,5} \cdot 100\% = 69,44\%. \]

Ответ: уровень логистического сервиса фирмы 69,44\%.