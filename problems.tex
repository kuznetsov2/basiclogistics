\section{Задачи}
\subsection{Задача 1}
На территории района имеется 6 магазинов.
Необходимо определить место расположения распределительного склада и нанести его на чертеж.
Исходные данные: координаты магазинов, грузооборот (таблица \ref{coord}).

\begin{table}[h!]
	\small
	\centering
	\caption{Координаты, грузооборот магазинов}
	\label{coord}
	\setlength{\extrarowheight}{1mm}
	\begin{tabularx}{\textwidth}{|p{3.73cm}|K{1.7cm}|K{1.7cm}|K{1.7cm}|K{1.7cm}|K{1.7cm}|K{1.7cm}|}
		\hline
		№ магазина      & 1  & 2  & 3  & 4  & 5  & 6  \\ \hline
		Координаты по Х & 15 & 25 & 40 & 50 & 80 & 90 \\ \hline
		Координаты по У & 35 & 45 & 70 & 80 & 40 & 60 \\ \hline
		Грузооборот, тонн/мес &10&35&25&40&45&60 \\ \hline
	\end{tabularx}
\end{table}

Координаты центра тяжести грузовых потоков ($X_{\text{склад}}, Y_{\text{склад}}$), то есть точки, в которых может быть размещен распределительный склад, определяются по формулам:
\[X_{\text{склад}}= \dfrac{\sum\limits_{i=1}^{n} \text{Г}_{i} \cdot X_{i}}{\sum\limits_{i=1}^{n} \text{Г}_{i}}, \]
\[Y_{\text{склад}}= \dfrac{\sum\limits_{i=1}^{n} \text{Г}_{i} \cdot Y_{i}}{\sum\limits_{i=1}^{n} \text{Г}_{i}} \]
Где $$\text{Г}_{i}$$ --- грузооборот i-го потребителя,
$X_{i}, Y_{i}$ --- координаты i-го потребителя,
n -- число потребителей.

\begin{figure}[h]
	\centering
	\includegraphics[width=1\linewidth]{problem1}
	\caption{Координаты магазинов с указанным грузооборотом}
	\label{fig:problem1}
\end{figure}
