\section{Автоматизированные методы распределения в коммерческой логистике}

С целью сокращения сроков производства продукции предприятия вкладывают значительные средства в автоматизацию своего производства.
Это позволяет значительно сократить время затрачиваемое на выпуск товаров: процессы, для осуществления которых требовалось несколько дней могут быть выполнены за несколько часов.
Разумно предположить, что время, затрачиваемое на доставку продукции потребителю возможно сократить подобным образом.
Для этого рассмотрим различные звенья цепи поставок, чтобы понять, как может быть сокращено время выполнения заказа.

%------

В последнее время все больше успешных  компаний используют информацию и информационные технологии для обеспечения быстрого реагирования на ситуацию на рыке.
Информационные системы изменяют форму организаций, а также характер отношений между ними.
Подкрепленная новыми технологиями информация становится движущей силой создания конкурентоспособной логистической стратегии.

Рассмотрим возникновение интегрированных логистических систем, связывающие различные функции предприятия.
Это могут быть производство и продвижение, с одной стороны, с операциями поставщика, а с другой --- с действиями покупателя.
Такие системы обозначаются общим названием систем планирования предприятия или планирования ресурсов предприятия (ERP).
Компании могут однозначно связывать пополнение количества товаров на рынке со своими операциями, происходящими <<выше по течению>>, и с операциями своих поставщиков посредством использования общей информации.
Использование подобны систем позволяет преобразовать каналы продвижения в каналы спроса в том смысле, что теперь система может реагировать на фактические потребности, а не на ожидаемую величину спроса, определенную на основе прогноза.

\begin{figure}[h]
	\centering
	\includegraphics[width=0.7\linewidth]{erp}
	\caption{Данные о ежедневных продажах используются для управления системой заказов на пополнение запасов}
	\label{fig:erp}
\end{figure}

Подобная информационная система используется компанией BhS, для управления процессом пополнения запасов одежды, продаваемой через свои английские магазины.
Ежедневное поступление данных из мест продаж позволяет руководителям фирмы определять потребности дополнительных поставок.
Информация напрямую передается поставщикам, которые упаковывают товары, заказываемые для конкретных магазинов, в специальные коробки, снабженные штрих-кодом.
Далее эти коробки собираются провайдером логистических услуг и поступают на перевалочную базу, также управляемую этим провайдером, где они рассортировываются и готовятся к отправке в магазины.
В результате обеспечивается поставка товаров <<точно в срок>>, что позволяет иметь в торговых точках минимум товарных запасов и сократить транспортные издержки за счет консолидации отправляемых грузов (см. рисунок \ref{fig:erp}).










