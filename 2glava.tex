\section{Автоматизированные методы распределения в коммерческой логистике}

С целью сокращения сроков производства продукции предприятия вкладывают значительные средства в автоматизацию своего производства.
Это позволяет значительно сократить время затрачиваемое на выпуск товаров: процессы, для осуществления которых требовалось несколько дней могут быть выполнены за несколько часов.
Разумно предположить, что время, затрачиваемое на доставку продукции потребителю возможно сократить подобным образом.
%Для этого рассмотрим различные звенья цепи поставок, чтобы понять, как может быть сокращено время выполнения заказа.

%------

В последнее время все больше успешных  компаний используют информацию и информационные технологии для обеспечения быстрого реагирования на ситуацию на рыке.
Информационные системы изменяют форму организаций, а также характер отношений между ними.
Подкрепленная новыми технологиями информация становится движущей силой создания конкурентоспособной логистической стратегии.

Рассмотрим возникновение интегрированных логистических систем, связывающие различные функции предприятия.
Это могут быть производство и продвижение, с одной стороны, с операциями поставщика, а с другой --- с действиями покупателя.
Такие системы обозначаются общим названием систем планирования предприятия или планирования ресурсов предприятия (ERP).
Компании могут однозначно связывать пополнение количества товаров на рынке со своими операциями, происходящими <<выше по течению>>, и с операциями своих поставщиков посредством использования общей информации.
Использование подобны систем позволяет преобразовать каналы продвижения в каналы спроса в том смысле, что теперь система может реагировать на фактические потребности, а не на ожидаемую величину спроса, определенную на основе прогноза \cite[с. 224]{christopher}.

\begin{figure}[bh]
	\centering
	\includegraphics[width=0.7\linewidth]{erp}
	\caption{Данные о ежедневных продажах используются для управления системой заказов на пополнение запасов}
	\label{fig:erp}
\end{figure}

Концепция ERP (Enterprise Resource Planning --- планирование ресурсов предприятия) предложена фирмой GartnerGroup в начале 1990-х.
Система управления предприятием, соответствующая концепции ERP, должна включать:
\begin{itemize}
	\item управление цепочкой поставок (Supply Chain Management --- SCM, ранее --- DRP, Distribution Resource Planning);
	\item усовершенствованное планирование и составление расписаний (Advanced Planning and Scheduling --- APS);
	\item модуль автоматизации продаж (Sales Force Automation --- SFA);
	\item автономный модуль, отвечающий за конфигурирование (Stand Alone Configuration Engine --- SACE);
	\item окончательное планирование ресурсов (Finite Resource Planning --- FRP);
	\item OLAP-технологии;
	\item модуль электронной коммерции (Electronic Commerce --- EC);
	\item управление данными об изделии (Poduct Data Management --- PDM).
\end{itemize}

Главная задача ERP-системы --- добиться оптимизации по времени и ресурсам всех перечисленных процессов.

Довольно часто вся присущая концепции ERP совокупност задач реализуется не одной интегрированной системой, а некоторым комплектом программного обеспечения.
В основе такого комплекта, как правило, лежит базовый ERP-пакет, к которому через соответствующие интерфейсы подключены специализированные продукты третьих фирм (отвечающие за электронную коммерцию, за OLAP, за автоматизацию продаж и проч.) \cite[с. 489]{grigoryev}.


Подобная информационная система используется компанией BhS, для управления процессом пополнения запасов одежды, продаваемой через свои английские магазины.
Ежедневное поступление данных из мест продаж позволяет руководителям фирмы определять потребности дополнительных поставок.
Информация напрямую передается поставщикам, которые упаковывают товары, заказываемые для конкретных магазинов, в специальные коробки, снабженные штрих-кодом.
Далее эти коробки собираются провайдером логистических услуг и поступают на перевалочную базу, также управляемую этим провайдером, где они рассортировываются и готовятся к отправке в магазины.
В результате обеспечивается поставка товаров <<точно в срок>>, что позволяет иметь в торговых точках минимум товарных запасов и сократить транспортные издержки за счет консолидации отправляемых грузов (см. рисунок \ref{fig:erp}).

В других случаях организации выясняют, что с помощью информации они могут управлять рассредоточенными запасами точно так же, как если бы все эти запасы находились в одном месте.
Выгоды от этого могут быть весьма значительными.
Если управление запасами носит централизованный характер и решения о пополнении ресурсов и о размерах заказов принимаются так, как если бы речь шла о едином запасе, то тогда требуется создание соответственно не нескольких, а только одного резервного запаса.
Сам запас может создаваться в любой точке системы --- либо вблизи места производства, либо вблизи места потребления.
В этом заключается суть концепции управления <<виртуальным>>, или, как его иногда называют, электронным, запасом.

Шведская компания SKF, производитель подшипников, создала европейскую информационную сеть, которую назвала глобальной системой прогнозирования и поставок (GFSS --- Global Forecasting and Supply System).
GFSS представляет собой систему управления спросом, которая определяет уровень спроса в каждый момент подачи заказа покупателями через локальные торговые отделения SKF или через систему электронного обмена данными (EDI).
Компьютерная система определяет в реальном времени место хранения требуемых товаров или, если эти товары в данный момент отсутствуют, заказ на них вносится в производственный график одного из пяти европейских заводов SKF.
В момент принятия заказа в GFSS, покупателю сообщается дата поставки требуемых подшипников, так как эта система параллельно разрабатывает и график транспортировки готовой продукции.

Такой тип систем получает все большее распространение, поскольку их архитектура позволяет легко обеспечить передачу данных с компьютера на компьютер.
Digital Equipment Corporation разработала общую интегрированную логистическую информационную систему, которая помогает организациям управлять спросом на всем протяжении логистического канала \cite[с. 225--228]{christopher}.








