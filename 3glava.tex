\section{Алгоритм формирования складской сети}

Формирование складской сети является проблемой стратегического уровня и решается, как правило, в связи с изменением стратегии компании и чаще всего как реакция на предложения маркетинга.
Особенно часто это происходит при изменении стратегий охвата рынка и выборе интенсивного распределения, а также при выходе на новые регионы сбыта или увеличении объема продаж.
В этом случае ориентиром распределения складских мощностей является определение службой маркетинга перспективных районов сбыта и объемов продаж в каждом регионе.

Основываясь на принципах системного подхода к анализу и синтезу логистических систем, технологию его применения к проблеме формирования складской сети можно представить в виде алгоритма (рисунок \ref{fig:algoritm}), который позволяет определить последовательность формирования складской сети, используемой компанией для достижения эффективного функционирования на рынке.


\begin{figure}[h]
	\centering
	\includegraphics[width=1\linewidth]{algoritm}
	\caption{Алгоритм формирования складской сети}
	\label{fig:algoritm}
\end{figure}

При формировании складской сети необходимо учитывать:
\begin{itemize}
	\item место складов в логистической системе компании;
	\item цели, задачи и функции компании по преобразованию вида и характеристик материального потока;
	\item территориальное расположение складской сети;
	\item взаимосвязи с внешней средой поставщиков и потребителей;
	\item характеристики используемых транспортных средств;
	\item состояние инфраструктуры сети;
	\item материально-техническую базу компании;
	\item наличие информационной связи внутри складской сети.
\end{itemize}

Алгоритм формирования складской сети в соответствии с методологией системного подхода должен быть конкретизирован применительно к цели исследования в виде комплекса соответствующих моделей, методов, организационно-технических и экономических разработок.

