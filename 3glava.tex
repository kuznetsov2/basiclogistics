\section{Алгоритм формирования складской сети}

Формирование складской сети является проблемой стратегического уровня и решается, как правило, в связи с изменением стратегии компании и чаще всего как реакция на предложения маркетинга.
Особенно часто это происходит при изменении стратегий охвата рынка и выборе интенсивного распределения, а также при выходе на новые регионы сбыта или увеличении объема продаж.
В этом случае ориентиром распределения складских мощностей является определение службой маркетинга перспективных районов сбыта и объемов продаж в каждом регионе.

Основываясь на принципах системного подхода к анализу и синтезу логистических систем, технологию его применения к проблеме формирования складской сети можно представить в виде алгоритма (рисунок \ref{fig:algoritm}), который позволяет определить последовательность формирования складской сети, используемой компанией для достижения эффективного функционирования на рынке.


При формировании складской сети необходимо учитывать:
\begin{itemize}
	\item место складов в логистической системе компании;
	\item цели, задачи и функции компании по преобразованию вида и характеристик материального потока;
	\item территориальное расположение складской сети;
	\item взаимосвязи с внешней средой поставщиков и потребителей;
	\item характеристики используемых транспортных средств;
	\item состояние инфраструктуры сети;
	\item материально-техническую базу компании;
	\item наличие информационной связи внутри складской сети.
\end{itemize}

Алгоритм формирования складской сети в соответствии с методологией системного подхода должен быть конкретизирован применительно к цели исследования в виде комплекса соответствующих моделей, методов, организационно-технических и экономических разработок.

\begin{figure}[h]
	\centering
	\includegraphics[width=1\linewidth]{algoritm}
	\caption{Алгоритм формирования складской сети}
	\label{fig:algoritm}
\end{figure}

Складская сеть создается или реорганизуется, в соответствии с маркетинговыми или логистическими стратегиями компании.
В связи с этим задача маркетинга заключается в определении границы рынка конкурентоспособности товара с учетом логистических затрат, включающих в первую очередь ориентировочную величину транспортных и складских издержек.
Логистика должна определить функциональное назначение складов и их задачи в рамках логистической системы с учетом целей компании.

Прогноз спроса --- первый этап формирования складской сети.
Методы прогнозирования должны отражать изменения, связанные с ориентацией на выполнение задач логистики.
При отборе таких методов прогнозирования следует произвести:
\begin{itemize}
	\item анализ ретроспективы спроса, опираясь на учет и анализ заказов, полученных фирмой ранее (в течение как можно более длительного срока);
	\item два параллельных вида работ: установить различие в видах и характере материальных потоков и, возможно, потребителей ,а также выбрать из всей гаммы существующих методов прогнозирования те, которые могут быть использованы в конкретных условиях;
	\item тестирование и отбор подходящих методов составления прогноза с учетом полученных данных о ретроспективе спроса и о типах товара; здесь проводится сопоставление полученных результатов с реальным спросом, зафиксированным в течение данного периода.
\end{itemize}

В заключение проводятся прогнозные расчеты и отслеживаются расхождения.
При этом постоянно оцениваются результаты функционирования системы прогнозирования.
При необходимости возможен пересмотр выбранных методов.

Планирование объема продаж и регионов сбыта осуществляется службой маркетинга на основании анализа рынка, сегментации рынков сбыта, анализа конкурентов и т. п.
Маркетинг берет на себя определение охвата рынка сбыта с учетом конкурентоспособной стоимости товара, включающей логистические издержки.
