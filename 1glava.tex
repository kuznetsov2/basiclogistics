\section{Логистические системы предприятия оптовой торговли, принципы их построения}

Система --- совокупность элементов, связанных друг с другом различными отношениями, образующих единое целое и противопоставляемая внешней среде.
Логистическую систему, как частный случай системы, можно охарактеризовать ее свойствами.
Отличительными особенностями логистических систем являются наличие потоковых процессов и системная целостность.

Рассмотрим свойства, которыми должен обладать объект, чтобы его можно было назвать логистической системой.

Целостность и членимость. Система --- это целостная совокупность элементов, взаимодействующих друг с другом.
Различают укрупненные структурные части логистической системы предприятия: персонал, склады, транспорт, транспортные пути, производственные участки.

Связи. Предприятия связаны коммерческими договорами, а подразделения предприятий --- производственными отношениями.

Организация. Необходимо создать упорядоченные связи между ее частями.

Интегративные качества. Наличие качеств, свойственных системе в целом, но не принадлежащих ни одному из ее звеньев в отдельности.
Служба снабжения может поставлять детали на предприятие, производственные участки обрабатывать детали, а отдел сбыта продавать готовую продукцию.
Но только предприятие в целом может обеспечивать потребителей нужными товарами.

В отечественной и зарубежной литературе различают два основных определения логистической системы.

Логистическая система --- это адаптивная система с обратной связью, выполняющая логистические функции и операции, состоящая из нескольких подсистем и имеющая развитые связи с внешней средой.

Данное определение относится к научному аспекту логистики, следующее же применимо к практической деятельности.

Логистическая система --- это сложная экономическая система, состоящая из звеньев и элементов, взаимосвязанных и объединенных единым процессом управления материальными и сопутствующими потоками для реализации корпоративной стратегии.

Логистическая система предприятия служит для продвижения материальных потоков, которое осуществляется квалифицированным персоналом с помощью разнообразной техники, технологий и специальных знаний \cite[с.114--116]{levkin}.

Логистика оптовых продаж, функционально, в рамках логистического цикла, представляет собой распределительную логистику.

Логистические системы предприятий оптовой торговли характеризуется следующими факторами:
\begin{itemize}
	\item широкий ассортиментный перечень реализуемых товаров;
	\item минимизация запасов у конечных потребителей, что требует осуществления частых поставок мелкими партиями;
	\item независимый спрос на товары, усложняющий управление запасами;
	\item повышение требования к гибкости обслуживания, связанное с возможностью незапланированных заказов;
	\item разветвленная складская сеть, максимально приближенная к потребителю;
	\item приоритет складской формы поставки перед транзитной для большинства видов товаров;
	\item сложность организации доставки конечным потребителям из-за наличия централизованной и децентрализованной систем поставок;
	\item значительное число посреднических структур.
\end{itemize}

Область распределения логистических систем предприятий оптовой торговли охватывает производителей  готовой продукции (сбыт начинается со складов готовой продукции), выпускающих товар на рынок сбыта, торговых посредников, логистических посредников, банки, страховые компании, предприятия сферы обслуживания и розничной торговли и т.д. 

Сложность системы распределения определяется разнообразием цепей поставок, включающих в себя каналы разных уровней.
Они предполагают активную посредническую деятельность, поглощающую до 80\% финансовых ресурсов фирм, и предъявляют высокие требования к координации и контролю над выполнением логистических операций.

Функционал логистики распределения включает следующие основные виды деятельности:
\begin{itemize}
	\item управление распределением материальных потоков при снабжении конечных потребителей;
	\item подготовка и поставка заказов в рамках функции управления заказами (логистическая составляющая);
	\item обслуживание клиентов посредством предоставления логистических услуг (логистический сервис);
	\item складирование и грузопереработка товарных запасов в сети распределения;
	\item транспортировка заказов потребителям;
	\item управление запасами в сети распределения;
	\item упаковка грузовой единицы в партии поставки;
	\item управление возвратом тары и дефектной продукции от потребителей.
\end{itemize}



















