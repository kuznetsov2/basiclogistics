\section{Логистические системы предприятия оптовой торговли, принципы их построения}

Понятие логистической системы является одним из фундаментальных понятий в логистике.
Так как логистическую систему можно рассматривать как частный случай по отношению к общему понятию системы, вначале рассмотрим общее определение понятия системы.
Затем можно будет определить, какие системы относят к классу логистических \cite[с. 82]{gadzhinskiy}.

Система --- совокупность элементов, связанных друг с другом различными отношениями, образующих единое целое и противопоставляемая внешней среде.
Логистическую систему, как частный случай системы, можно охарактеризовать ее свойствами.
Отличительными особенностями логистических систем являются наличие потоковых процессов и системная целостность.

Рассмотрим свойства, которыми должен обладать объект, чтобы его можно было назвать логистической системой.

Целостность и членимость. Система --- это целостная совокупность элементов, взаимодействующих друг с другом.
Различают укрупненные структурные части логистической системы предприятия: персонал, склады, транспорт, транспортные пути, производственные участки.

Связи. Предприятия связаны коммерческими договорами, а подразделения предприятий --- производственными отношениями.

Организация. Необходимо создать упорядоченные связи между ее частями.

Интегративные качества. Наличие качеств, свойственных системе в целом, но не принадлежащих ни одному из ее звеньев в отдельности.
Служба снабжения может поставлять детали на предприятие, производственные участки обрабатывать детали, а отдел сбыта продавать готовую продукцию.
Но только предприятие в целом может обеспечивать потребителей нужными товарами \cites[с. 114-115]{levkin}[с. 83]{gadzhinskiy}.

В процессе продвижения материальных потоков участвует квалифицированный персонал, транспортные средства, погрузочно-разгрузочные устройства, различные здания и сооружения.
Ход процесса зависит от того, в какой степени подготовлены к нему сами движущиеся и периодически накапливаемые в запасах грузы.
Совокупность производительных сил, обеспечивающих прохождение грузов, всегда как-то организована.
Так, если имеются материальные потоки, то всегда присутствует какая-то товаропроводящая система.
Традиционно, эти системы специально не проектируются, а возникают как результат деятеьности отдельных элементов.
Это различные предприятия, или подразделения одного предприятия.

Логистика ставит и решает задачу проектирования гармоничных, согласованных логистических систем, с заданными параметрами материальных потоков на выходе.
Эти системы отличает высокая степень согласованности входящих в них производительных сил в вопросах управления сквозными материальными потоками.

Рассмотрим свойства логистических систем в разрезе каждого из четырех свойств, присущих любой системе, которые были приведены выше.

1. Система есть целостная совокупность элементов, взаимодействующих друг с другом.
Разложение логистических систем на элементы можно проводить различными способами.
На макроуровне при прохождении материального потока от одного предприятия к другому в качестве элементов могут рассматриваться сами эти предприятия, а также связывающий их транспорт.

На микроуровне логистическая система может быть представлена в виде следующих основных подсистем:
\begin{itemize}
	\item Закупка --- подсистема, которая обеспечивает поступление материального потока в логистическую систему.
	\item Планирование и управление производством --- эта подсистема принимает материальный поток  от подсистемы закупок и управляет им в процессе выполнения различных технологических операций, превращающих предмет труда в продукт труда.
	\item Сбыт --- подсистема, которая обеспечивает выбытие материального потока из логистической системы.
\end{itemize}

Таким образом, элементы логистических систем обладают разными качествами, при этом совместимы друг с другом.
Совместимость обеспечивается единством цели, которой подчинено функционирование каждого элемента системы.

2. Связи.
Между элементами логистической системы имеются существенные связи, которые с закономерной необходимостью определяют интегративные качества.
В макрологистических системах основу связи между элементами составляет договор.
В микрологистических системах элементы связаны производственными отношениями.

3. Организация.
Связи между элементами логистической системы определенным образом упорядочены, т. е. логистическая система имеет организацию.

4. Интегративные качества.
Логистическая система обладает интегративными качествами, не свойственными ни одному из элементов в отдельности.
Это способность поставить нужный товар в нужное время, в нужное место, необходимого качества, с минимальными затратами, а также способность адаптироваться к изменяющимся условиям внешней среды (изменение спроса на товар или услуги, непредвиденный выход из строя технических средств и т. п.).

Интегративные качества логистической системы позволяют ей закупать материалы, пропускать их через свои производственные мощности и выдавать во внешнюю среду, достигая при этом заранее намеченных целей.

Логистическую систему, способную ответить на возни­кающий спрос быстрой поставкой нужного товара, можно сравнить с живым организмом.
Мускулы этого организма --- подъемно-транспортная техника, центральная нервная си­стема --- сеть компьютеров на рабочих местах участников логистического процесса, организованная в единую инфор­мационную систему.
По размерам этот организм может за­нимать территорию завода или предприятие оптовой тор­говли, а может охватывать регион или выходить за преде­лы государства.
Он способен адаптироваться, приспосабли­ваться к возмущениям внешней среды, реагировать на нее в том же темпе, в котором происходят события\cite[с.84--87]{gadzhinskiy}.

В отечественной и зарубежной литературе различают два основных определения логистической системы.
В  отечественной литературе распространение получило следующее:

\begin{quote}
	Логистическая система --- это адаптивная система с обратной связью, выполняющая логистические функции и операции, состоящая из нескольких подсистем и имеющая развитые связи с внешней средой\cite.
\end{quote}

Данное определение относится к научному аспекту логистики, следующее же применимо к практической деятельности.

Логистическая система --- это сложная экономическая система, состоящая из звеньев и элементов, взаимосвязанных и объединенных единым процессом управления материальными и сопутствующими потоками для реализации корпоративной стратегии.

Логистическая система предприятия служит для продвижения материальных потоков, которое осуществляется квалифицированным персоналом с помощью разнообразной техники, технологий и специальных знаний \cite[с.116]{levkin}.


%%%%%%%%%%%%%

Логистика оптовых продаж, функционально, в рамках логистического цикла, представляет собой распределительную логистику.

Логистические системы предприятий оптовой торговли характеризуется следующими факторами:
\begin{itemize}
	\item широкий ассортиментный перечень реализуемых товаров;
	\item минимизация запасов у конечных потребителей, что требует осуществления частых поставок мелкими партиями;
	\item независимый спрос на товары, усложняющий управление запасами;
	\item повышение требования к гибкости обслуживания, связанное с возможностью незапланированных заказов;
	\item разветвленная складская сеть, максимально приближенная к потребителю;
	\item приоритет складской формы поставки перед транзитной для большинства видов товаров;
	\item сложность организации доставки конечным потребителям из-за наличия централизованной и децентрализованной систем поставок;
	\item значительное число посреднических структур.
\end{itemize}

Область распределения логистических систем предприятий оптовой торговли охватывает производителей  готовой продукции (сбыт начинается со складов готовой продукции), выпускающих товар на рынок сбыта, торговых посредников, логистических посредников, банки, страховые компании, предприятия сферы обслуживания и розничной торговли и т.д. 

Сложность системы распределения определяется разнообразием цепей поставок, включающих в себя каналы разных уровней.
Они предполагают активную посредническую деятельность, поглощающую до 80\% финансовых ресурсов фирм, и предъявляют высокие требования к координации и контролю над выполнением логистических операций.

Функционал логистики распределения включает следующие основные виды деятельности:
\begin{itemize}
	\item управление распределением материальных потоков при снабжении конечных потребителей;
	\item подготовка и поставка заказов в рамках функции управления заказами (логистическая составляющая);
	\item обслуживание клиентов посредством предоставления логистических услуг (логистический сервис);
	\item складирование и грузопереработка товарных запасов в сети распределения;
	\item транспортировка заказов потребителям;
	\item управление запасами в сети распределения;
	\item упаковка грузовой единицы в партии поставки;
	\item управление возвратом тары и дефектной продукции от потребителей.
\end{itemize}

В логистике распределения важную роль играет сбытовая деятельность компании.
Она предполагает организацию сбытовой системы для осуществления продажи продукции.

Сбытовая система может быть представлена как совокупность субъектов рынка, участвующих в товарно-денежных отношениях с компанией, занимающейся реализацией готовой продукции.

Выбор рациональной системы сбыта по регионам решается службой маркетинга в рамках стратегии охвата потребителей.
Реализация службой маркетинга стратегий охвата  потребителей и рационального выбора сбытовой системы требуют от нее координации со службой логистики.
Совместная деятельность логистики и маркетинга при формировании сбытовой системы сводится к следующему: маркетинг на основе проведенного анализа рынка потребителей, анализа конкурентной среды и определения потенциальных регионов сбыта определяет структуру потребителей по типам клиентов и планирует объем продаж.
Анализ конкурентной среды в регионах потенциального сбыта и востребованности услуг со стороны клиентской базы позволяет маркетингу установить основные требования к обслуживанию потребителей и определить конкурентоспособные цены.
Однако рациональный выбор сбытовой системы в каждом регионе сбыта маркетинг может сделать только на основе анализа затрат, связанных в первую очередь с деятельностью необходимой логистической инфраструктуры и предоставлением требуемого уровня обслуживания.
Подробный анализ альтернативных затрат всех возможных вариантов сбытовых систем осуществляет логистика.
Эти затраты и возможность осуществления логистикой требуемого уровня обслуживания и являются основными показателями при выборе маркетингом альтернативной сбытовой системы.

Выделяют три основных типа сбытовых систем, основанных:
\begin{enumerate}
	\item [1)] на развитии собственной сбытовой сети;
	\item [2)] развитии дилерской сети (связанная сбытовая система);
	\item [3)] сбыте товара независимому оптовику (независимая сбытовая система).
\end{enumerate}
\cite[с. 251--254 ]{dybskaya}
















