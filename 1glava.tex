\section{Логистические системы предприятия оптовой торговли, принципы их построения}
\subsection{1}
Теза́урус, в общем смысле — специальная терминология, более строго и предметно — словарь, собрание сведений, корпус или свод, полномерно охватывающие понятия, определения и термины специальной области знаний или сферы деятельности, что должно способствовать правильной лексической, корпоративной коммуникации; в современной лингвистике — особая разновидность словарей, в которых указаны семантические отношения между лексическими единицами. Тезаурусы являются одним из действенных инструментов для описания отдельных предметных областей.

алфавит морской ёж был колюч 
there was a lot of candies everywhere