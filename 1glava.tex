\section{Логистические системы предприятия оптовой торговли, принципы их построения}

Система --- совокупность элементов, связанных друг с другом различными отношениями, образующих единое целое и противопоставляемая внешней среде.
Логистическую систему, как частный случай системы, можно охарактеризовать ее свойствами.
Отличительными особенностями логистических систем являются наличие потоковых процессов и системная целостность.

Рассмотрим свойства, которыми должен обладать объект, чтобы его можно было назвать логистической системой.

Целостность и членимость. Система --- это целостная совокупность элементов, взаимодействующих друг с другом.
Различают укрупненные структурные части логистической системы предприятия: персонал, склады, транспорт, транспортные пути, производственные участки.

Связи. Предприятия связаны коммерческими договорами, а подразделения предприятий --- производственными отношениями.

Организация. Необходимо создать упорядоченные связи между ее частями.

Интегративные качества. Наличие качеств, свойственных системе в целом, но не принадлежащих ни одному из ее звеньев в отдельности.
Служба снабжения может поставлять детали на предприятие, производственные участки обрабатывать детали, а отдел сбыта продавать готовую продукцию.
Но только предприятие в целом может обеспечивать потребителей нужными товарами.

В отечественной и зарубежной литературе различают два основных определения логистической системы.

Логистическая система --- это адаптивная система с обратной связью, выполняющая логистические функции и операции, состоящая из нескольких подсистем и имеющая развитые связи с внешней средой.

Данное определение относится к научному аспекту логистики, следующее же применимо к практической деятельности.

Логистическая система --- это сложная экономическая система, состоящая из звеньев и элементов, взаимосвязанных и объединенных единым процессом управления материальными и сопутствующими потоками для реализации корпоративной стратегии.

Логистическая система предприятия служит для продвижения материальных потоков, которое осуществляется квалифицированным персоналом с помощью разнообразной техники, технологий и специальных знаний \cite[с.114--116]{levkin}.

